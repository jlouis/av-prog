%% This file is my default inclusion of packages I need when writing
%% things in LaTeX. The list here is pretty comprehensive, but sets up
%% the main stuff I use in all projects I start on.

%% This part sets up the font basics. I like mathpazo a lot. It sets
%% up Palatino as a clone with pazo math. It is much more beautiful
%% than many other fonts you can choose from. I also like some of the
%% mathdesign variants, in particular [utopia,osf]{mathdesign} with no
%% linespread. We could make it a setting here.
%%
%% Also, add microtype. We can't really live without it.
\usepackage{microtype}
\usepackage[T1]{fontenc}
\usepackage{lmodern}
\usepackage[final]{listings}
\lstset{ %
language=Octave,                % choose the language of the code
basicstyle=\footnotesize,       % the size of the fonts that are used for the code
numbers=left,                   % where to put the line-numbers
numberstyle=\footnotesize,      % the size of the fonts that are used for the line-numbers
stepnumber=1,                   % the step between two line-numbers. If it's 1 each line will be numbered
numbersep=5pt,                  % how far the line-numbers are from the code
%backgroundcolor=\color{white},  % choose the background color. You must add \usepackage{color}
showspaces=false,               % show spaces adding particular underscores
showstringspaces=true,         % underline spaces within strings
showtabs=true,                 % show tabs within strings adding particular underscores
frame=single,	                % adds a frame around the code
tabsize=2,	                % sets default tabsize to 2 spaces
captionpos=b,                   % sets the caption-position to bottom
breaklines=true,                % sets automatic line breaking
breakatwhitespace=true,        % sets if automatic breaks should only happen at whitespace
escapeinside={\%*}{*)}          % if you want to add a comment within your code
}

%\usepackage[osf]{mathpazo}
%\linespread{1.05}
%\usepackage[sc,osf,utopia]{mathdesign}

%% AMS gives us the basic math-support
\usepackage{amsmath}

%% AMS Theorem support. We set up a good list of theorems and make
%% sure that they are numbered appropriately in the order of main
%% theorems. I hate when a document contains too many counters so this
%% keeps the number of live counters down to a minimum.
\usepackage{amsthm}
\theoremstyle{plain}
\newtheorem{axm}{Axiom}
\newtheorem{thm}{Theorem}
\newtheorem{lem}[thm]{Lemma}
\newtheorem{prop}[thm]{Proposition}
\newtheorem*{cor}{Corollary}

\theoremstyle{definition}
\newtheorem{defn}[thm]{Definition}
\newtheorem{conj}[thm]{Conjecture}
\newtheorem{exmp}[thm]{Example}
\theoremstyle{remark}
\newtheorem*{rem}{Remark}
\newtheorem*{note}{Note}
\newtheorem{case}{Case}


%% Hyperref should be last
\usepackage{hyperref}
