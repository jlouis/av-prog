\section{Compiler Alternative}
\label{chap:compiler-alter}

Late in the project, due to the problems with compiling the zeroCheck,
we discovered a smarter compilation scheme of SEL to 2D. Recall, that
the compiler introduced in Section \ref{compiler} works by recursively
compiling sub expressions. However, a module in 2D has continuation
semantics. A use of a module define a continuation point where the
program will continue once the program returns. Thus, one can
CPS-transform SEL and output each part of the CPS transformed
program. It turns out this is surprisingly simple, though a
disadvantage is the excessive use of modules.

\begin{figure}
  \begin{quote}
\begin{verbatim}
,...................................,
: zerocont3214                      :
:                                   :
---------+                          :
:        v                          :
: *============*  *================*:
: !case[N] E S !->!use "zcase3526" !-
: *============*  *================*:
:      	 |        *================*:
:        +------->!use "scase6543" !-
:                 *================*:
,...................................,
\end{verbatim}
  \end{quote}
  \caption{Case compilation example}
  \label{fig:2}
\end{figure}

As an example, consider the compilation of a zero control. In Figure
\ref{fig:2}, we present the continuation-compilation of the zero
control. The Monad we need for this contain a writer transformer so we
can output modules as we go (\texttt{MonadState s (WriterT w m)}). The
control is compiled as a hardcoded module where the values of the
``uses'' can be replaced. The pseudocode for compilation is then:

\begin{samepage}
\begin{verbatim}
compileE (ZCont test zcase scase) cid =
  do
    zcase_id <- compileE zcase cid
    scase_id <- compileE scase cid
    zerocont_id <- new
    tell $ mkZerocont zerocont_id zcase_id scase_id
    test_id <- compileE test zerocont_id
    return test_id
\end{verbatim}
\end{samepage}
where the \texttt{mkZerocont} function outputs the hardcoded box with
the identifiers substituted.

We believe this compilation scheme is much easier to implement that
the one we tried to implement. The key observation is that
continuation passing style fits 2D directly. We believe that all other
SEL-constructs are as easy to compile.

Unfortunately time prevents us from implementing this idea even though
we estimate it would take a couple of days at most. We also estimate
that this can be done in less than 150 lines of code. Comparing this
to our current implementation of above 1000 lines, this is a gain of
considerable size.

\paragraph{Optimization}
\label{sec:optimization}

In the above example, one does not even need to take care of 2D layout
at all. One can simply output all the modules. Optimization is
possible with a layouter however. One can easily implement an
optimizer by inlining modules if the module always returns to the same
continuation. This optimization will reify as large modules as
possible and is fairly easy to program.


%%% Local Variables:
%%% mode: latex
%%% TeX-master: "master"
%%% End:
